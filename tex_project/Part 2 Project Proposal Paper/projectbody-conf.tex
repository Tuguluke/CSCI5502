\section{ Problem Statement/motivation}
(like homework 1, what knowledge and how would you apply that knowledge, what is interesting that you hope to find)
	There are many reports saying US is (or will) face a technology workers (Technologists). Our study will mine through the US H1B data set along with Dice.com's job and candidate profiles (along with skill sets for each job title), to try to understand which technologist job are in increasing/decreasing demands, which skill setsa are more popular/unpupular, through time and geo-location analysis. Our goal is to better understand the US technology market and what we can do to help people who try to get into tech market by providing them with skillset guidance. 

\section{Literature survey }
(previous work) describe and cite.
\section{Proposed Work }
E.g., what do you need to do for data collection, preprocessing (cleaning  integrating, transforming, etc.), process for derived data, design, evaluation. Describe how it is different than what has been done previously from your literature survey (or if replicating).
\begin{itemize}
	\item Data scraping: Screaping HTML data from the web and store them in seperate json file
	\item Data cleaning: Probably the most important part of the project
	\item Data preprocessing: 
	\item Data integration: Integrate internal data with mined H1b data 
	\item Data mining and analysis
\end{itemize}
\section{Data set }
(make sure you have the data set!). Provide URL and details about the data set (similar to homework 1, chapter 2, etc.)
	\begin{itemize}
	\item 
	
	H1B data  scraped from  I scraped from open data online.\footnote{\boxed{https://h1bdata.info/index.php}.}
	The data set is mainly from the United States Department of Labor (DOL) on how many H1-B petitions were filed and approved, with detailed information such as Employer, the title of the job, city/state locations, along with base salary, and submission and acceptance dates. 
	\item Skill set (Internal data)
	\item US Bureau of Labor Statistics \\
	\footnote{\boxed{https://www.bls.gov/developers/api_python.htm}
}
	\item Data mined and stored on my machine
\end{itemize}
\section{ Evaluation Methods }
E.g., metrics, existing solutions, …\\
here are many ways we can evaluate the results. Comparing  time series analysis data within each job we can compute the increaing or decreasing sides of each job. Location analysis on which job are located where with repect to other jobs and locations. 

\section{Tools}
\begin{itemize}
	\item Beautifulsoup for mining
	\item Python library for all the computing and analysis
	\item JSON for data store
\end{itemize}
\section{Milestones}
What you plan to have done by when\\
The Following table is my work plan for carrying out the project.
\begin{center}
	\begin{tabular}{ | l |p{5cm} |}
		\hline
		Date & Work Plan \\ \hline
		Week 1 & Submit the Research Proposal.\\\hline
		Week 1 & Submit the Research Proposal.\\\hline
		Week 1 & Submit the Research Proposal.\\\hline
		Week 1 & Submit the Research Proposal.\\\hline
		Week 1 & Submit the Research Proposal.\\\hline
		Week 1 & Submit the Research Proposal.\\\hline
		Week 1 & Submit the Research Proposal.\\\hline
		Week 1 & Submit the Research Proposal.\\\hline
		Week 1 & Submit the Research Proposal.\\\hline
		Week 1 & Submit the Research Proposal.\\\hline
	\end{tabular}
\end{center}










